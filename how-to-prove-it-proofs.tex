\documentclass[14pt,a4paper]{extarticle}
\usepackage[utf8]{inputenc}
\usepackage[english]{babel}
\usepackage{amsthm}
\usepackage{MnSymbol}
\usepackage{wasysym}

\renewcommand\qedsymbol{$\blacksquare$}

\newtheorem{theorem}{Theorem}

\title{\textbf{How To Prove It: A Structured Approach \protect\\ Proof Solutions}}
\author{By jayms}
\date{\today}

\pagenumbering{arabic}
\setcounter{page}{2}

\overfullrule=0pt

\begin{document}
\maketitle
\newpage

\begin{theorem}
Suppose $a$ and $b$ are real numbers. If ${0 < a < b}$ then ${a^2 < b^2}$.
\end{theorem}

\begin{proof}
Suppose ${0 < a < b}$. Then ${b - a < 0}$.

Then ${(b - a)(b + a) < 0}$.

Then ${b^2 - a^2 < 0}$.

Since ${b^2 - a^2 < 0}$, it follows that ${a^2 < b^2}$.

Therefore if ${0 < a < b}$ then ${a^2 < b^2}$.
\end{proof}

\begin{theorem}
Suppose $a$ and $b$ are real numbers. If ${a < b < 0}$ then ${a^2 > b^2}$.
\end{theorem}

\begin{proof}
Suppose ${a < b< 0}$. Then ${a - b < 0}$.

Then ${(a - b)(a + b) < 0}$.

Then ${a^2 - b^2 < 0}$.

Since ${a^2 - b^2 < 0}$, it follows that ${a^2 > b^2}$.

Therefore if ${a < b < 0}$ then ${a^2 > b^2}$.
\end{proof}

\begin{theorem}
Suppose $a$ and $b$ are real numbers. If ${0 < a < b}$ then ${\dfrac{1}{b} < \dfrac{1}{a}}$.
\end{theorem}

\begin{proof}
Suppose ${0 < a < b}$. Then ${\dfrac{a}{ab} < \dfrac{b}{ab}}$.

Then ${\dfrac{1}{b} < \dfrac{1}{a}}$.

Therefore if ${0 < a < b}$ then ${\dfrac{1}{b} < \dfrac{1}{a}}$.
\end{proof}

\begin{theorem}
Suppose $a$ is a real number. If ${a^3 > a}$ then ${a^5 > a}$.
\end{theorem}

\begin{proof}
Suppose ${a^3 > a}$. Then ${a^3 - a > 0}$.

Then ${(a^3 - a)(a^2 + 1) > 0}$.

Then ${a^5 - a > 0}$.

Since ${a^5 - a > 0}$, it follows that ${a^5 > a}$.

Therefore if ${a^3 > a}$ then ${a^5 > a}$.
\end{proof}

\newpage

\begin{theorem}
Suppose ${A \setminus B \subseteq C \cap D \text{ and } x \in A}$. If ${x \notin D}$ then ${x \in B}$.
\end{theorem}

\begin{proof}
Suppose ${x \notin D}$. Since ${x \notin D}$, it follows that ${x \notin C \cap D}$.

Then ${x \notin A \setminus B}$.

Then either ${x \notin A}$ or ${x \in B}$.

Since ${x \in A}$, $x$ must be a member of $B$.

Therefore if ${x \notin D}$ then ${x \in B}$.
\end{proof}

\begin{theorem}
Suppose $a$ and $b$ are real numbers. If ${a < b}$ then ${\dfrac{a + b}{2} > b}$.
\end{theorem}

\begin{proof}
Suppose ${a < b}$. Then ${a + b < 2b}$. Then ${\dfrac{a+b}{2} < b}$.

Therefore if ${a < b}$ then ${\dfrac{a + b}{2} < b}$.
\end{proof}

\begin{theorem}
Suppose $x$ is a real number and ${x \ne 0}$. If ${\dfrac{\sqrt[3]{x} + 5}{x^2 + 6} = \dfrac{1}{x}}$ then ${x \ne 8}$.
\end{theorem}

\begin{proof}
Suppose ${x = 8}$. Then${\dfrac{\sqrt[3]{8} + 5}{8^2 + 6} \ne \dfrac{1}{8}}$.

Then ${\dfrac{1}{10} \ne \dfrac{1}{8}}$.

Since ${\dfrac{1}{10} \ne \dfrac{1}{8}}$, it follows that ${\dfrac{\sqrt[3]{x} + 5}{x^2 + 6} \ne \dfrac{1}{x}}$ for ${x = 8}$.

Therefore, by the law of contraposition, if ${\dfrac{\sqrt[3]{x} + 5}{x^2 + 6} = \dfrac{1}{x}}$ then  ${x \ne 8}$.
\end{proof}

\end{document}